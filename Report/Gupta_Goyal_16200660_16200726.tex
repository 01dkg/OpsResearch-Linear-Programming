\documentclass[paper=a4, fontsize=11pt]{scrartcl} % A4 paper and 10pt font size
\usepackage[T1]{fontenc} % Use 8-bit encoding that has 256 glyphs
%\usepackage{fourier} % Use the Adobe Utopia font for the document - comment this line to return to the LaTeX default
\usepackage[english]{babel} % English language/hyphenation
\usepackage{lscape}
\usepackage{mathptmx}
\usepackage{url}
\usepackage{sectsty} % Allows customizing section commands
\setlength{\columnsep}{5mm}
\usepackage{natbib}
\usepackage{setspace}
\usepackage{lipsum}
\usepackage{amsmath}
\usepackage{relsize}
\usepackage[most]{tcolorbox}
\usepackage{inconsolata}
 \usepackage{booktabs}
\newtcblisting[auto counter]{sexylisting}[2][]{sharp corners, 
    fonttitle=\bfseries, colframe=gray, listing only, 
    listing options={basicstyle=\ttfamily,language=R}, 
    title=Listing \thetcbcounter: #2, #1}
\title{MIS40520 Analytics Business Modelling - Assignment 1}

\DeclareOldFontCommand{\rm}{\normalfont\rmfamily}{\mathrm}
\DeclareOldFontCommand{\sf}{\normalfont\sffamily}{\mathsf}
\DeclareOldFontCommand{\tt}{\normalfont\ttfamily}{\mathtt}
\DeclareOldFontCommand{\bf}{\normalfont\bfseries}{\mathbf}
\DeclareOldFontCommand{\it}{\normalfont\itshape}{\mathit}
\DeclareOldFontCommand{\sl}{\normalfont\slshape}{\@nomath\sl}
\DeclareOldFontCommand{\sc}{\normalfont\scshape}{\@nomath\sc}
\DeclareRobustCommand*\cal{\@fontswitch\relax\mathcal}
\DeclareRobustCommand*\mit{\@fontswitch\relax\mathnormal}
%----------------------------------------------------------------------------------------
%	Work Break Down Sturture of Analytics type model
%----------------------------------------------------------------------------------------
\usepackage{tikz}
\usetikzlibrary{arrows,shapes,positioning,shadows,trees}
\tikzset{
every node/.style={draw,text width=2cm,drop shadow},
style1/.style= {rectangle, rounded corners=2pt, thin,align=center,fill=green!30},
style2/.style= {rectangle, rounded corners=6pt, thin,align=center,fill=yellow!40},
style3/.style= {rectangle,thin,align=left,fill=pink!60,text width=2.7cm, text height=1.6mm, font=\small}
}
%----------------------------------------------------------------------------------------
%	Caption Set & URL Font Control
%----------------------------------------------------------------------------------------
\usepackage[font=footnotesize,labelfont=bf]{caption}
\captionsetup{justification=centering}
\urlstyle{rm}
\usepackage{authblk}

\usepackage{geometry}
 \geometry{
 a4paper,
 total={170mm,257mm},
 left=20mm,
 top=20mm,
 footskip=10mm
 }

\usepackage{graphicx}
\graphicspath{ {images/} }
\usepackage{indentfirst}
\usepackage{dirtree}
\setlength\parindent{0pt} % Removes all indentation from paragraphs - comment this line for an assignment with lots of text

%----------------------------------------------------------------------------------------
%	Header Footer
%----------------------------------------------------------------------------------------
\usepackage{fancyvrb}
\usepackage{fancyhdr}
\pagestyle{fancy}
\lhead{MIS40520: Analytical Business Modelling}
\rhead{Shruti Goyal \\ Deepak Kumar Gupta}
\cfoot{\thepage}

%----------------------------------------------------------------------------------------
%	TITLE SECTION
%----------------------------------------------------------------------------------------
\begin{document}
\begin{titlepage}
\newcommand{\HRule}{\rule{\linewidth}{0.5mm}} % Defines a new command for the horizontal lines, change thickness here
\center % Center everything on the page

%----------------------------------------------------------------------------------------
%	HEADING SECTIONS
%----------------------------------------------------------------------------------------
\textsc{\LARGE UCD Michael Smurfit Graduate Business School}\\[1.7cm] % Name of your university/college
\includegraphics[scale = 0.7]{logo.png} \\ [1cm]
\textrm{\Large	\textrm{MIS40520: Analytical Business Modelling}}\\[0.6cm] % Major heading such as course name

%----------------------------------------------------------------------------------------
%	TITLE SECTION
%----------------------------------------------------------------------------------------

\HRule \\[0.5cm]
{ \LARGE  \textrm{M(I)LP Assignment 2}}\\[0.5cm] % Title of your document
\HRule \\[1.5cm]
 
%----------------------------------------------------------------------------------------
%	AUTHOR SECTION
%----------------------------------------------------------------------------------------

\begin{minipage}{0.45\textwidth}
\begin{flushleft} \large
\emph{\textrm{Authors:}}\\
\normalsize{Goyal \textrm{Shruti} (16200726)}\\
\normalsize{Gupta \textrm{Deepak Kumar} (16200660) }\\
\end{flushleft}
\end{minipage}
~
~
\begin{minipage}{0.45\textwidth}
\begin{flushright} \large
\emph{\textrm{Lecturer:}} \\
\normalsize{Dr Paula \textrm{Carroll}} % Supervisor's Name
\end{flushright}
\end{minipage}\\[3cm]
% If you don't want a supervisor, uncomment the two lines below and remove the section above
%\Large \emph{Author:}\\
%John \textsc{Smith}\\[3cm] % Your name

%----------------------------------------------------------------------------------------
%	DATE SECTION
%----------------------------------------------------------------------------------------
\vspace{3cm}
{\large \today}\\[1cm] % Date, change the \today to a set date if you want to be precise
\vfill % Fill the rest of the page with whitespace
\end{titlepage}

%% Begin Writing the Document

%----------------------------------------------------------------------------------------
%	MAIN BODY
%----------------------------------------------------------------------------------------
%\setlength{\parindent}{10ex}
%Context

\pagenumbering{roman}
\tableofcontents
\listoffigures
\listoftables
\addcontentsline{toc}{section}{Executive Summary}
\cleardoublepage
\clearpage

%----------------------------------------------------------------------------------------
%	EXECUTIVE SUMMARY
%----------------------------------------------------------------------------------------
\subtitle{\normalsize{Literature Review: Big data Analytics for Healthcare}}
\title{Executive Summary}
\author{\small{Shruti Goyal, Deepak Kumar Gupta and Dr James McDermott}}
\date{}
\maketitle
\textbf{Objective:} To review big data analytics in healthcare.
\\
\\
\textbf{Methods:} The review describes big data analytics and its implementation in healthcare, highlights gaps in previous studies describes analytical platforms and examples, discusses the existing challenges, offers a conclusion and, states an open question for future work.
\\
\\
\textbf{Results:} The review provides the overview of advancements and changes in big data analytics for 
practitioners.
\\
\\
\textbf{Conclusion:} Big data analytics implementation in healthcare has seen a tremendous growth over past 6 years. However, there are few challenges which still need to be addressed.
\\ 
\\
\textbf{Keywords:} Healthcare, Data Analytics, Clinics, Systematic Review, Tools and Techniques
\clearpage	
%----------------------------------------------------------------------------------------
%	ABSTRACT SECTION
%----------------------------------------------------------------------------------------
\begin{abstract}
\title{Abstract}
Yayayayay
\end{abstract}
\newpage 
\pagenumbering{arabic}

%----------------------------------------------------------------------------------------
%	BACKGROUND
%----------------------------------------------------------------------------------------
\section{Background}
You have been asked to help organise hotel accommodation for 150 delegates
attending a conference in UCD. You use a hotel booking website to extract the data
shown in the Table below for ten four star hotels in Dublin city centre. The Table
shows the price for a single room per person per night, the customer satisfaction
rating and the number of rooms available for each hotel. The conference organisers
want to allocate the delegates to hotels at minimum cost while achieving an average
customer satisfaction rate of at least 8.3. Formulate an (I)LP model of this problem. 

\begin{table}[ht]
\centering
\begin{tabular}{@{}|l|l|l|l|l|l|l|l|l|l|l|@{}}
\toprule
Hotel Index       & 1   & 2   & 3   & 4   & 5   & 6   & 7   & 8   & 9   & 10  \\ \midrule
Price (euro)     & 89  & 99  & 119 & 112 & 143 & 94  & 130 & 98  & 155 & 152 \\ \midrule
Customer rating   & 7.8 & 8.3 & 8   & 8.7 & 8   & 8.1 & 8.6 & 8.9 & 8.9 & 8.4 \\ \midrule
Room Availability & 35  & 30  & 15  & 15  & 15  & 20  & 15  & 10  & 10  & 20  \\ \bottomrule
\end{tabular}
\caption{Hotel Data}
\label{hoteldata}
\end{table}
%----------------------------------------------------------------------------------------
%	METHODOLOGY
%----------------------------------------------------------------------------------------


%----------------------------------------------------------------------------------------
%	INTRODUCTION
%----------------------------------------------------------------------------------------

\section{Introduction}

Discussion about research papers
\\
\begin{figure}[ht]
\centering
	\includegraphics[scale = 0.7]{logo.png} 
	\caption{Four V's of Big Data}
	\caption*{Source: \url{http://www.datasciencecentral.com/profiles/blogs/data-veracity}}	
\end{figure}

then discuss about integer programming take some example from internet to demostrate
\section{Methodology}

We have total delegates 150 and 10 four star hotels to accommodate guests
We need to minimize the cost while achieving minimum average rating:
\\
Lets say $HI$ as $Hotel$ $Index$:
\\

Objective is: 
\begin{equation}
minimize{(Cost)}
\end{equation}
Constraint 1: 
\begin{equation}
‎‎Cost‎ ‎=‎ \mathlarger{\mathlarger{‎‎\sum}}_{HI=1}^{10}{Price(HI)* decVar(HI}
\end{equation}

Constraint 2: 
\begin{equation}
Avg. Rating ‎=‎ \frac{\mathlarger{\mathlarger{‎‎\sum}}_{HI=1}^{10}{Rating(HI)*decVar(HI))}}{150}
\end{equation}

Constraint 3: 
\begin{equation}
decVar(HI) <= Avail(HI), HI \in \{1,10\}
\end{equation}



%----------------------------------------------------------------------------------------
%	RESULTS AND ANALYSIS
%----------------------------------------------------------------------------------------

\section{Analysis of Results}

Compare non integer and integer solutions

Why interger programming is required

Also take previous assignment example to contrast the need on MILP

%----------------------------------------------------------------------------------------
%	CONCLUSIONS
%----------------------------------------------------------------------------------------

\section{Conclusions}

\appendix
\section{Code}
\begin{sexylisting}{Craft Brewer: LP Model}
model Conference_Management
uses "mmxprs"; !gain access to the Xpress-Optimizer solver
	declarations
		HI = 1..10                       	! Index range
		PRICE: array(HI) of integer         ! Price table
		RATING: array(HI) of real			!Customer Satsifcation rating
		AVAIL: array(HI) of integer			!Available Rooms in a hotel
		decVar: array(HI) of mpvar          !Decision Variables
	end-declarations
	Max_Guest := 150
	
	!Read in the data from our text file
	initializations from 'Conference_Management.txt'
		PRICE
		RATING
		AVAIL
	end-initializations
	
	!procedure to check problem status
	procedure print_status
		declarations
			status: string
		end-declarations
		case getprobstat of
		XPRS_OPT: status:="LP Optimum found"
		XPRS_UNF: status:="Unfinished"
		XPRS_INF: status:="Infeasible"
		XPRS_UNB: status:="Unbounded"
		XPRS_OTH: status:="Failed"
		else status:="???"
		end-case
		writeln("Problem status: ", status)
	end-procedure
	 
	!Minimiz cost
	Cost:= sum(i in HI) PRICE(i)*decVar(i)
	 
	!Constraints
	!Declare that our decision variables are integers
	forall (i in HI) do 
		decVar(i) is_integer
	end-do
	
	AvgRating:= (sum(i in HI) RATING(i)*decVar(i))/150 >= 8.3
			 
	forall( i in HI)
		decVar(i) <= AVAIL(i)	 	
	sum(i in HI) decVar(i)	= Max_Guest
		
	!Display output of solution values
	procedure print_sol	
		writeln("Begin running model") 	
		writeln("---------------------------------------------------------")
		print_status
		writeln("Cost is: €",getobjval)
		writeln("---------------------------------------------------------")
		forall(i in HI) 
			writeln("Passenger in Hotel_",i," -> ", getsol(decVar(i)))
			
		!write value of AvgRating to output
		writeln(getsol(AvgRating))
		writeln("---------------------------------------------------------")
		exportprob(EP_MIN,"",Cost)
		writeln("---------------------------------------------------------")
		exportprob(1,"Conference_Management",Cost)
		writeln("End running model")
		!Modify Optimizer control parameter PSEUDOCOST
	end-procedure 
	minimize(Cost)
	print_sol	
end-model
\end{sexylisting}
\section{Input and Output}
\subsection{Case 1: Original Data (10 Variables)}
\textbf{Input}
\begin{verbatim}
! Data file for `20. Conference Management Assignment - 2'

PRICE: [ 89, 99, 119, 112,143,94,130,98,155,152] !Constraint coefficients	 	

RATING: [7.8,8.3,8.0,8.7,8.0,8.1,8.6,8.9,8.9,8.4] !Constraint coefficients

AVAIL:[35,30,15,15,15,20,15,10,10,20] !Values of the constraints
\end{verbatim}
\textbf{Output}
\begin{verbatim}
Begin running model
---------------------------------------------------------
Problem status: LP Optimum found
Cost is: €16207
---------------------------------------------------------
Passenger in Hotel_1 -> 35
Passenger in Hotel_2 -> 30
Passenger in Hotel_3 -> 6
Passenger in Hotel_4 -> 15
Passenger in Hotel_5 -> 0
Passenger in Hotel_6 -> 20
Passenger in Hotel_7 -> 15
Passenger in Hotel_8 -> 10
Passenger in Hotel_9 -> 10
Passenger in Hotel_10 -> 9
0.000666667
---------------------------------------------------------
\ Using Xpress-MP extensions
Minimize
 89 decVar(1) + 99 decVar(2) + 119 decVar(3) + 112 decVar(4) + 143 decVar(5) + 
94 decVar(6) + 130 decVar(7) + 98 decVar(8) + 155 decVar(9) + 152 decVar(10)

Subject To
_R1: decVar(1) + decVar(2) + decVar(3) + decVar(4) + decVar(5) + decVar(6) + 
decVar(7) + decVar(8) + decVar(9) + decVar(10) = 150
AvgRating: 0.052 decVar(1) + 0.0553333 decVar(2) + 0.0533333 decVar(3) + 0.058 decVar(4) + 
0.0533333 decVar(5) + 0.054 decVar(6) + 0.0573333 decVar(7) + 0.0593333 decVar(8) + 
0.0593333 decVar(9) + 0.056 decVar(10) >= 8.3

Bounds
decVar(1) <= 35
decVar(2) <= 30
decVar(3) <= 15
decVar(4) <= 15
decVar(5) <= 15
decVar(6) <= 20
decVar(7) <= 15
decVar(8) <= 10
decVar(9) <= 10
decVar(10) <= 20

Integers
decVar(1) decVar(2) decVar(3) decVar(4) decVar(5) decVar(6) decVar(7) decVar(8) 
decVar(9) decVar(10) 

End
---------------------------------------------------------
End running model
\end{verbatim}

\subsection{Case 2: Scale Up to 20 Variables}
\textbf{Input}
\begin{verbatim}
! Data file for `20. Conference Management Assignment - 2'

PRICE: [ 89, 99, 119, 112,143,94,130,98,155,152,55,68,99,140,87,75,78,110,96,74] !Constraint coefficients	 	

RATING: [7.8,8.3,8.0,8.7,8.0,8.1,8.6,8.9,8.9,8.4,9.5,7.6,9.5,7.9,8.3,8.2,9.9,8.8,8.0,8.9] !Constraint coefficients

AVAIL:[35,30,15,15,15,20,15,10,10,20,2,5,6,7,9,1,2,5,4,6] !Values of the constraints
\end{verbatim}
\textbf{Output}
\begin{verbatim}
Begin running model
---------------------------------------------------------
Problem status: LP Optimum found
Cost is: €14061
---------------------------------------------------------
Passenger in Hotel_1 -> 35
Passenger in Hotel_2 -> 30
Passenger in Hotel_3 -> 0
Passenger in Hotel_4 -> 15
Passenger in Hotel_5 -> 0
Passenger in Hotel_6 -> 20
Passenger in Hotel_7 -> 0
Passenger in Hotel_8 -> 10
Passenger in Hotel_9 -> 0
Passenger in Hotel_10 -> 0
Passenger in Hotel_11 -> 2
Passenger in Hotel_12 -> 5
Passenger in Hotel_13 -> 6
Passenger in Hotel_14 -> 0
Passenger in Hotel_15 -> 9
Passenger in Hotel_16 -> 1
Passenger in Hotel_17 -> 2
Passenger in Hotel_18 -> 5
Passenger in Hotel_19 -> 4
Passenger in Hotel_20 -> 6
0.0306667
---------------------------------------------------------
\ Using Xpress-MP extensions
Minimize
 89 decVar(1) + 99 decVar(2) + 119 decVar(3) + 112 decVar(4) + 143 decVar(5) + 
94 decVar(6) + 130 decVar(7) + 98 decVar(8) + 155 decVar(9) + 152 decVar(10) + 
55 decVar(11) + 68 decVar(12) + 99 decVar(13) + 140 decVar(14) + 87 decVar(15) + 
75 decVar(16) + 78 decVar(17) + 110 decVar(18) + 96 decVar(19) + 74 decVar(20)

Subject To
_R1: decVar(1) + decVar(2) + decVar(3) + decVar(4) + decVar(5) + decVar(6) + 
decVar(7) + decVar(8) + decVar(9) + decVar(10) + decVar(11) + decVar(12) + decVar(13) + 
decVar(14) + decVar(15) + decVar(16) + decVar(17) + decVar(18) + decVar(19) + 
decVar(20) = 150
AvgRating: 0.052 decVar(1) + 0.0553333 decVar(2) + 0.0533333 decVar(3) + 0.058 decVar(4) + 
0.0533333 decVar(5) + 0.054 decVar(6) + 0.0573333 decVar(7) + 0.0593333 decVar(8) + 
0.0593333 decVar(9) + 0.056 decVar(10) + 0.0633333 decVar(11) + 0.0506667 decVar(12) + 
0.0633333 decVar(13) + 0.0526667 decVar(14) + 0.0553333 decVar(15) + 0.0546667 decVar(16) + 
0.066 decVar(17) + 0.0586667 decVar(18) + 0.0533333 decVar(19) + 0.0593333 decVar(20) >= 8.3

Bounds
decVar(1) <= 35
decVar(2) <= 30
decVar(3) <= 15
decVar(4) <= 15
decVar(5) <= 15
decVar(6) <= 20
decVar(7) <= 15
decVar(8) <= 10
decVar(9) <= 10
decVar(10) <= 20
decVar(11) <= 2
decVar(12) <= 5
decVar(13) <= 6
decVar(14) <= 7
decVar(15) <= 9
decVar(17) <= 2
decVar(18) <= 5
decVar(19) <= 4
decVar(20) <= 6

Integers
decVar(1) decVar(2) decVar(3) decVar(4) decVar(5) decVar(6) decVar(7) decVar(8) 
decVar(9) decVar(10) decVar(11) decVar(12) decVar(13) decVar(14) decVar(15) 
decVar(16) decVar(17) decVar(18) decVar(19) decVar(20) 

End
---------------------------------------------------------
End running model
\end{verbatim}



\subsection{Case 2: Scale Up to 20 Variables}
\textbf{Input}
\begin{verbatim}
! Data file for `20. Conference Management Assignment 2'

PRICE: [ 89, 99, 119, 112,143,94,130,98,155,152,55,68,99,140,87,75,78,110,96,74,65,5,40,30,30,17,15,89,10,20] !Constraint coefficients	 	

RATING: [7.8,8.3,8.0,8.7,8.0,8.1,8.6,8.9,8.9,8.4,9.5,7.6,9.5,7.9,8.3,8.2,9.9,8.8,8.0,8.9,8.1,8.3,8.2,8.5,8.9,9.5,7.8,9.1,8.1,8.2] !Constraint coefficients

AVAIL:[35,30,15,15,15,20,15,10,10,20,2,5,6,7,9,1,2,5,4,6,2,5,4,6,1,2,3,17,2,5] !Values of the constraints
\end{verbatim}
\textbf{Output}
\begin{verbatim}
Begin running model
---------------------------------------------------------
Problem status: LP Optimum found
Cost is: €11395
---------------------------------------------------------
Passenger in Hotel_1 -> 35
Passenger in Hotel_2 -> 3
Passenger in Hotel_3 -> 0
Passenger in Hotel_4 -> 0
Passenger in Hotel_5 -> 0
Passenger in Hotel_6 -> 20
Passenger in Hotel_7 -> 0
Passenger in Hotel_8 -> 10
Passenger in Hotel_9 -> 0
Passenger in Hotel_10 -> 0
Passenger in Hotel_11 -> 2
Passenger in Hotel_12 -> 5
Passenger in Hotel_13 -> 6
Passenger in Hotel_14 -> 0
Passenger in Hotel_15 -> 9
Passenger in Hotel_16 -> 1
Passenger in Hotel_17 -> 2
Passenger in Hotel_18 -> 0
Passenger in Hotel_19 -> 4
Passenger in Hotel_20 -> 6
Passenger in Hotel_21 -> 2
Passenger in Hotel_22 -> 5
Passenger in Hotel_23 -> 4
Passenger in Hotel_24 -> 6
Passenger in Hotel_25 -> 1
Passenger in Hotel_26 -> 2
Passenger in Hotel_27 -> 3
Passenger in Hotel_28 -> 17
Passenger in Hotel_29 -> 2
Passenger in Hotel_30 -> 5
0.0713333
---------------------------------------------------------
\ Using Xpress-MP extensions
Minimize
 89 decVar(1) + 99 decVar(2) + 119 decVar(3) + 112 decVar(4) + 143 decVar(5) + 
94 decVar(6) + 130 decVar(7) + 98 decVar(8) + 155 decVar(9) + 152 decVar(10) + 
55 decVar(11) + 68 decVar(12) + 99 decVar(13) + 140 decVar(14) + 87 decVar(15) + 
75 decVar(16) + 78 decVar(17) + 110 decVar(18) + 96 decVar(19) + 74 decVar(20) + 
65 decVar(21) + 5 decVar(22) + 40 decVar(23) + 30 decVar(24) + 30 decVar(25) + 
17 decVar(26) + 15 decVar(27) + 89 decVar(28) + 10 decVar(29) + 20 decVar(30)

Subject To
_R1: decVar(1) + decVar(2) + decVar(3) + decVar(4) + decVar(5) + decVar(6) + 
decVar(7) + decVar(8) + decVar(9) + decVar(10) + decVar(11) + decVar(12) + decVar(13) + 
decVar(14) + decVar(15) + decVar(16) + decVar(17) + decVar(18) + decVar(19) + 
decVar(20) + decVar(21) + decVar(22) + decVar(23) + decVar(24) + decVar(25) + 
decVar(26) + decVar(27) + decVar(28) + decVar(29) + decVar(30) = 150
AvgRating: 0.052 decVar(1) + 0.0553333 decVar(2) + 0.0533333 decVar(3) + 0.058 decVar(4) + 
0.0533333 decVar(5) + 0.054 decVar(6) + 0.0573333 decVar(7) + 0.0593333 decVar(8) + 
0.0593333 decVar(9) + 0.056 decVar(10) + 0.0633333 decVar(11) + 0.0506667 decVar(12) + 
0.0633333 decVar(13) + 0.0526667 decVar(14) + 0.0553333 decVar(15) + 0.0546667 decVar(16) + 
0.066 decVar(17) + 0.0586667 decVar(18) + 0.0533333 decVar(19) + 0.0593333 decVar(20) + 
0.054 decVar(21) + 0.0553333 decVar(22) + 0.0546667 decVar(23) + 0.0566667 decVar(24) + 
0.0593333 decVar(25) + 0.0633333 decVar(26) + 0.052 decVar(27) + 0.0606667 decVar(28) + 
0.054 decVar(29) + 0.0546667 decVar(30) >= 8.3

Bounds
decVar(1) <= 35
decVar(2) <= 30
decVar(3) <= 15
decVar(4) <= 15
decVar(5) <= 15
decVar(6) <= 20
decVar(7) <= 15
decVar(8) <= 10
decVar(9) <= 10
decVar(10) <= 20
decVar(11) <= 2
decVar(12) <= 5
decVar(13) <= 6
decVar(14) <= 7
decVar(15) <= 9
decVar(17) <= 2
decVar(18) <= 5
decVar(19) <= 4
decVar(20) <= 6
decVar(21) <= 2
decVar(22) <= 5
decVar(23) <= 4
decVar(24) <= 6
decVar(26) <= 2
decVar(27) <= 3
decVar(28) <= 17
decVar(29) <= 2
decVar(30) <= 5

Integers
decVar(1) decVar(2) decVar(3) decVar(4) decVar(5) decVar(6) decVar(7) decVar(8) 
decVar(9) decVar(10) decVar(11) decVar(12) decVar(13) decVar(14) decVar(15) 
decVar(16) decVar(17) decVar(18) decVar(19) decVar(20) decVar(21) decVar(22) 
decVar(23) decVar(24) decVar(25) decVar(26) decVar(27) decVar(28) decVar(29) 
decVar(30) 

End
---------------------------------------------------------
End running model

\end{verbatim}
\bibliographystyle{agsm}
\addcontentsline{toc}{section}{References} 
\end{document}

