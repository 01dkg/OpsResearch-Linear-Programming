\documentclass[paper=a4, fontsize=11pt]{scrartcl} % A4 paper and 10pt font size
\usepackage[T1]{fontenc} % Use 8-bit encoding that has 256 glyphs
%\usepackage{fourier} % Use the Adobe Utopia font for the document - comment this line to return to the LaTeX default
\usepackage[english]{babel} % English language/hyphenation
\usepackage{lscape}
\usepackage{mathptmx}
\usepackage{url}
\usepackage{sectsty} % Allows customizing section commands
\setlength{\columnsep}{5mm}
\usepackage{natbib}
\usepackage{setspace}
\DeclareOldFontCommand{\rm}{\normalfont\rmfamily}{\mathrm}
\DeclareOldFontCommand{\sf}{\normalfont\sffamily}{\mathsf}
\DeclareOldFontCommand{\tt}{\normalfont\ttfamily}{\mathtt}
\DeclareOldFontCommand{\bf}{\normalfont\bfseries}{\mathbf}
\DeclareOldFontCommand{\it}{\normalfont\itshape}{\mathit}
\DeclareOldFontCommand{\sl}{\normalfont\slshape}{\@nomath\sl}
\DeclareOldFontCommand{\sc}{\normalfont\scshape}{\@nomath\sc}
\DeclareRobustCommand*\cal{\@fontswitch\relax\mathcal}
\DeclareRobustCommand*\mit{\@fontswitch\relax\mathnormal}
%----------------------------------------------------------------------------------------
%	Work Break Down Sturture of Analytics type model
%----------------------------------------------------------------------------------------
\usepackage{tikz}
\usetikzlibrary{arrows,shapes,positioning,shadows,trees}
\tikzset{
every node/.style={draw,text width=2cm,drop shadow},
style1/.style= {rectangle, rounded corners=2pt, thin,align=center,fill=green!30},
style2/.style= {rectangle, rounded corners=6pt, thin,align=center,fill=yellow!40},
style3/.style= {rectangle,thin,align=left,fill=pink!60,text width=2.7cm, text height=1.6mm, font=\small}
}
%----------------------------------------------------------------------------------------
%	Caption Set & URL Font Control
%----------------------------------------------------------------------------------------
\usepackage[font=footnotesize,labelfont=bf]{caption}
\captionsetup{justification=centering}
\urlstyle{rm}
\usepackage{authblk}

\usepackage{geometry}
 \geometry{
 a4paper,
 total={170mm,257mm},
 left=20mm,
 top=20mm,
 footskip=10mm
 }

\usepackage{graphicx}
\graphicspath{ {images/} }
\usepackage{indentfirst}
\usepackage{dirtree}
\setlength\parindent{0pt} % Removes all indentation from paragraphs - comment this line for an assignment with lots of text

%----------------------------------------------------------------------------------------
%	Header Footer
%----------------------------------------------------------------------------------------
\usepackage{fancyvrb}
\usepackage{fancyhdr}
\pagestyle{fancy}
\lhead{MIS40520: Analytical Business Modelling}
\rhead{Shruti Goyal \\ Deepak Kumar Gupta}
\cfoot{\thepage}

%----------------------------------------------------------------------------------------
%	TITLE SECTION
%----------------------------------------------------------------------------------------
\begin{document}
\begin{titlepage}
\newcommand{\HRule}{\rule{\linewidth}{0.5mm}} % Defines a new command for the horizontal lines, change thickness here
\center % Center everything on the page

%----------------------------------------------------------------------------------------
%	HEADING SECTIONS
%----------------------------------------------------------------------------------------
\textsc{\LARGE UCD Michael Smurfit Graduate Business School}\\[1.7cm] % Name of your university/college
\includegraphics[scale = 0.7]{logo.png} \\ [1cm]
\textrm{\Large	\textrm{MIS40520: Analytical Business Modelling}}\\[0.6cm] % Major heading such as course name

%----------------------------------------------------------------------------------------
%	TITLE SECTION
%----------------------------------------------------------------------------------------

\HRule \\[0.5cm]
{ \LARGE  \textrm{M(I)LP Assignment 2}}\\[0.5cm] % Title of your document
\HRule \\[1.5cm]
 
%----------------------------------------------------------------------------------------
%	AUTHOR SECTION
%----------------------------------------------------------------------------------------

\begin{minipage}{0.45\textwidth}
\begin{flushleft} \large
\emph{\textrm{Authors:}}\\
\normalsize{Goyal \textrm{Shruti} (16200726)}\\
\normalsize{Gupta \textrm{Deepak Kumar} (16200660) }\\
\end{flushleft}
\end{minipage}
~
~
\begin{minipage}{0.45\textwidth}
\begin{flushright} \large
\emph{\textrm{Lecturer:}} \\
\normalsize{Dr Paula \textrm{Carroll}} % Supervisor's Name
\end{flushright}
\end{minipage}\\[3cm]
% If you don't want a supervisor, uncomment the two lines below and remove the section above
%\Large \emph{Author:}\\
%John \textsc{Smith}\\[3cm] % Your name

%----------------------------------------------------------------------------------------
%	DATE SECTION
%----------------------------------------------------------------------------------------
\vspace{3cm}
{\large \today}\\[1cm] % Date, change the \today to a set date if you want to be precise
\vfill % Fill the rest of the page with whitespace
\end{titlepage}

%% Begin Writing the Document

%----------------------------------------------------------------------------------------
%	MAIN BODY
%----------------------------------------------------------------------------------------
%\setlength{\parindent}{10ex}
%Context

\pagenumbering{roman}
\tableofcontents
\cleardoublepage
\addcontentsline{toc}{section}{List of Figures}
\listoffigures
\cleardoublepage
\addcontentsline{toc}{section}{List of Tables}
\listoftables
\cleardoublepage
\addcontentsline{toc}{section}{Abstract}
\cleardoublepage
\clearpage

%----------------------------------------------------------------------------------------
%	ABSTRACT SECTION
%----------------------------------------------------------------------------------------
\subtitle{\normalsize{Literature Review: Big data Analytics for Healthcare}}
\title{Abstract}
\author{\small{Shruti Goyal, Deepak Kumar Gupta and Dr James McDermott}}
\date{}
\maketitle
\textbf{Objective:} To review big data analytics in healthcare.
\\
\\
\textbf{Methods:} The review describes big data analytics and its implementation in healthcare, highlights gaps in previous studies describes analytical platforms and examples, discusses the existing challenges, offers a conclusion and, states an open question for future work.
\\
\\
\textbf{Results:} The review provides the overview of advancements and changes in big data analytics for 
practitioners.
\\
\\
\textbf{Conclusion:} Big data analytics implementation in healthcare has seen a tremendous growth over past 6 years. However, there are few challenges which still need to be addressed.
\\ 
\\
\textbf{Keywords:} Healthcare, Data Analytics, Clinics, Systematic Review, Tools and Techniques
\begin{abstract}
\end{abstract}
\newpage 
\pagenumbering{arabic}
%----------------------------------------------------------------------------------------
%	INTRODUCTION
%----------------------------------------------------------------------------------------

\section{Introduction}
Big Data refers to the large and complex datasets (structured or unstructured) that overwhelmed a business on daily basis, which traditional database management system could not be processed. Although, this concept is constantly changing with the everyday rapid explosion of data." Various attempts are defining big data essentially categorized it as a collection of data elements whose size, speed, type and or complexity require one to seek, adopt and invent new hardware and software mechanisms in order to successfully store, analyze ad visualize the data." \citep{belle2015big}
\\
\\
\textbf{Four V's of Big Data}
\\
To understand what exactly big data is, one need to understand the four V's of Big Data \citep{zikopoulos2012harness}:
\\
\begin{figure}[ht]
\centering
	\includegraphics[scale = 0.7]{logo.png} 
	\caption{Four V's of Big Data}
	\caption*{Source: \url{http://www.datasciencecentral.com/profiles/blogs/data-veracity}}	
\end{figure}
\appendix
\section{\\Title of Appendix A}
% the \\ insures the section title is centered below the phrase: AppendixA

Text of Appendix A is Here
\bibliographystyle{agsm}
\addcontentsline{toc}{section}{References} 
\end{document}

